\documentclass[10pt, a4paper]{article}
\usepackage{fontspec} 
\usepackage{fontawesome}
\usepackage[russian]{babel}

% DOCUMENT LAYOUT
\usepackage{geometry} 
\geometry{a4paper, textwidth=5.5in, textheight=8.5in, marginparsep=7pt, marginparwidth=.6in}
\setlength\parindent{0in}

% FONTS
\usepackage[usenames,dvipsnames]{color}
\usepackage{xunicode}
\usepackage{xltxtra}
\defaultfontfeatures{Mapping=tex-text}
\setromanfont [Ligatures={Common}, Numbers={OldStyle}, Variant=01]{Linux Libertine O}
\setmonofont[Scale=0.8]{Monaco}

% ---- CUSTOM COMMANDS
\chardef\&="E050
\newcommand{\html}[1]{\href{#1}{\scriptsize\textsc{[html]}}}
\newcommand{\pdf}[1]{\href{#1}{\scriptsize\textsc{[pdf]}}}
\newcommand{\doi}[1]{\href{#1}{\scriptsize\textsc{[doi]}}}
% ---- MARGIN YEARS
\usepackage{marginnote}
\newcommand{\amper{}}{\chardef\amper="E0BD }
\newcommand{\years}[1]{\marginnote{\scriptsize #1}}
\renewcommand*{\raggedleftmarginnote}{}
\setlength{\marginparsep}{7pt}
\reversemarginpar

\usepackage{sectsty} 
\usepackage[normalem]{ulem} 
\sectionfont{\mdseries\upshape\Large}
\subsectionfont{\mdseries\scshape\normalsize} 
\subsubsectionfont{\mdseries\upshape\large} 

\def\myauthor{Фоат Ахмадеев}
\def\mytitle{\myauthor - резюме}
\def\myemail{foat.akhmadeev@gmail.com}
\def\mygithub{github.com/foat}
\def\mylinkedin{linkedin.com/in/akhmadeevfoat}
\def\myhomesite{foat.me}

\usepackage[bookmarks, colorlinks, breaklinks, 
	pdftitle={\myauthor - vita},
	pdfauthor={\myauthor}
]{hyperref}  
\hypersetup{linkcolor=blue,citecolor=blue,filecolor=black,urlcolor=MidnightBlue} 


% DOCUMENT
\begin{document}
{\LARGE \myauthor}\\[1cm]
\begin{minipage}[t]{2in}
\flushleft {Java/C++ программист,\\Разработчик в области компьютерного зрения}
\end{minipage}
\begin{minipage}[t]{4in}
\flushleft
\faHome \, \href{http://\myhomesite}{\myhomesite}\\
\faEnvelopeAlt \, \href{mailto:\myemail}{\myemail}\\
\faGithub \, \href{https://\mygithub}{\mygithub}\\
\faLinkedin \, \href{https://\mylinkedin}{\mylinkedin}\\
\end{minipage}

\section*{Интересы}
Управление проектами • Компьютерное зрение • Разработка программного обеспечения\\
Веб разработка • Электронная коммерция • Писательская деятельность

\section*{Опыт работы}
\noindent
\years{2015-по настоящее время}{Фриланс, \textbf{Java/C++ программист,\\Разработчик в области компьютерного зрения}
\begin{itemize}
\item{Java/C++ разработка.}
\item{Исследование и разработка в области компьютерного зрения.}
\item{Общение с иностранными заказчиками.}
\item{Проекты:
\begin{itemize}
\item{Java фреймворк для фильтрации пользователей.}
\item{Распознавание дорожной разметки на C++.}
\item{Детектирование нескольких объектов по одному шаблону на изображении для Android.}
\item{Обновление \href{https://github.com/contrasetup/contra}{интерпретатора}, написанного на Java.}
\end{itemize}
}
\item{Основные технологии: Java, JavaScript, C++, Android, MySQL, Spring Framework, GAE, OpenCV, Build automation tools, Git.}
\end{itemize}
\years{2014-2015}{\href{http://cmp.felk.cvut.cz}{CMP}, \textbf{Исследователь в области компьютерного зрения.}
\begin{itemize}
\item{Исследование и разработка в области компьютерного зрения.}
\item{Получение 3D информации об объекте с 2D изображения.}
\item{Проект: Image rectification using vanishing lines and local affine frames.}
\item{Основные технологии: C++, Matlab, Unix, Git.}
\end{itemize}
\years{2013-2014}{\href{http://roadar.ru}{RoadAR}, \textbf{Разработчик в области компьютерного зрения}
\begin{itemize}
\item{Исследование и разработка в области компьютерного зрения.}
\item{Распознавание дорожных знаков в реальном времени на телефоне.}
\item{Проект: \href{https://play.google.com/store/apps/details?id=ru.roadar.android}{RoadAR}}.
\item{Основные технологии: C++, Java, OpenCV, PostgreSQL, Maven, Android NDK, Boost library, CMake, Git.}
\end{itemize}
}
\years{2013}{\href{http://dz.ru}{Digital Zone}, \textbf{Старший Java разработчик.}
\begin{itemize}
\item{Управление проектами и проверка кода.}
\item{Проведение технических интервью.}
\item{Java веб frontend и backend разработка.}
\item{Проект: \href{http://www.ulmart.ru}{ulmart.ru} (\href{http://dz.ru/portfolio/clients/ulmart/release_1-0/}{link 1}, \href{http://dz.ru/portfolio/clients/ulmart/release_2-0/}{link 2}).}
\item{Основные технологии: Java, MySQL, Spring MVC, Apache Solr, JavaScript, Application servers, Freemarker, Git, Maven.}
\end{itemize}
}
\years{2011-2013}{\href{http://dz.ru}{Digital Zone}, \textbf{Java разработчик}
\begin{itemize}
\item{Java веб frontend и backend разработка.}
\item{Проекты:
\begin{itemize}
\item{Сайты связанные с социальными сетями.}
\item{\href{http://www.ulmart.ru}{ulmart.ru}}.
\end{itemize}
}
\item{Основные технологии: Java, GWT, Spring MVC, EJB, Apache Solr, JavaScript, Application servers, Freemarker, Version control, Build automation tools.}
\end{itemize}
}

\section*{Образование}
\noindent
\years{2012-2014}{\textbf{Магистр информационных технологий}\\
\textit{Казанский федеральный университет}, Россия\\
\textsc{руководитель}: Евгений Столов, Казанский федеральный университет
\begin{itemize}
\item{Компьютерное зрение, Обработка изображений.}
\item{Проект: \textit{Восстановление 3D модели урбанистического пейзажа с одного изображения}. Публикация: \cite{foat2014prediction}}.
\item{Основные технологии: Matlab, C++, \TeX}.
\end{itemize}
}
\years{2008-2012}{\textbf{Бакалавр информационных технологий}\\
\textit{Казанский федеральный университет}, Россия\\
\textsc{руководители}: Евгений Столов, Казанский федеральный университет; Александр Шлянников, Казанский федеральный университет
\begin{itemize}
\item{Компьютерное зрение, Обработка изображений, Веб разработка.}
\item{Проекты: \textit{Распознавание рукописных цифр}, \textit{Электронная библиотека с использованием GWT}}.
\item{Основные технологии: Java, JEE, GWT}.
\end{itemize}
}

\section*{Навыки в области ИТ}
Основные языки (Java, C++).\\
Анализ данных (Matlab).\\
Скриптовые языки (Shell script, JavaScript).\\
Языки разметки (XML, JSON, HTML, CSS).\\
Языки запросов (SQL).\\
Сервера приложений (JBoss, Apache Tomcat, Jetty).\\
Системы управления версиями (Git, Subversion).\\
Системы сборки приложений (Maven, Gradle, CMake).\\
Операционные системы (OS X, Linux, Windows).\\
Другое (Spring MVC, EJB, GWT, JQuery, MySQL, PostgreSQL, GAE).\\
Компьютерная верстка (\TeX, \LaTeX).

\section*{Знание языков}
\textit{Русский} (родной)\\
\textit{Английский} (свободное владение)

\renewcommand{\refname}{\section*{Публикации}\subsection*{Опубликованные работы}}
\bibliographystyle{abbrv}
\bibliography{lib}

\vfill{}

\begin{center}
{\scriptsize  Последнее обновление: \today\- •\- 
% ---- PLEASE LEAVE THIS BACKLINK FOR ATTRIBUTION AS PER CC-LICENSE
Создано с помощью \href{http://nitens.org/taraborelli/cvtex}{
\fontspec{Times New Roman}Lua\TeX}\\
\href{http://foat.me/cv-ru.pdf}{http://foat.me/cv-ru.pdf}}
\end{center}

\end{document}